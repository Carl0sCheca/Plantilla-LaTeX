% Cambia el título de la tabla de contenido
% \renewcommand{\contentsname}{Índice}

% Compatibilidad con babel y tikz
\usetikzlibrary{babel}

% Ruta de las imágenes
\graphicspath{{./images/}}

% Número de página en landscape
\def\fillandplacepagenumber{%
 \par\pagestyle{empty}%
 \vbox to 0pt{\vss}\vfill
 \vbox to 0pt{\baselineskip0pt
   \hbox to\linewidth{\hss}%
   \baselineskip\footskip
   \hbox to\linewidth{%
     \hfil\thepage\hfil}\vss}}

% Número de página en landscape pdf insertado con pagestyle personalizado
\fancypagestyle{mylandscape}{%
  \fancyhf{}% Clear header/footer
  \fancyfoot{% Footer
    \makebox[\textwidth][r]{% Right
      \rlap{\hspace{\footskip}% Push out of margin by \footskip
        \smash{% Remove vertical height
          \raisebox{\dimexpr.5\baselineskip+\footskip+.6\textheight}{% Raise vertically
            \rotatebox{90}{\thepage}}}}}}% Rotate counter-clockwise
  \renewcommand{\headrulewidth}{0pt}% No header rule
  \renewcommand{\footrulewidth}{0pt}% No footer rule
}


% Ocultar sección
\newcommand\invisiblesection[1]{%
  \refstepcounter{section}%
  \addcontentsline{toc}{section}{\protect\numberline{\thesection}#1}%
  \sectionmark{#1}}

% Ocultar subsección
\newcommand\invisiblesubsection[1]{%
\refstepcounter{subsection}%
\addcontentsline{toc}{subsection}{\protect\numberline{\thesubsection}#1}%
\sectionmark{#1}}

% Insertar pdf en formato apaisado
% \begin{landscape}
%     \includepdf[pagecommand={\thispagestyle{mylandscape}}, noautoscale=true, scale=.85, pages={-}, turn=false, landscape]{NOMBREPDF.pdf}
% \end{landscape}

% Insertar imagen
% \begin{figure}[H]
%     \centering
%     \includegraphics[trim={0 0 0 0}, clip, width=1\linewidth]{NOMBREIMAGEN.png}
%     \footnotesize PIE DE IMAGEN
% \end{figure}

% Texto con color de fondo
% \colorbox{gray!30}{texto}

% Cabecera personalizada, usar con \pagestyle{custom1} o \thispagestyle{custom1}
\fancypagestyle{custom1}{%
  \fancyhf{} % Limpia las cabeceras y los pie de página incluída la página
  \rhead{\color{gray}\writeTitle}
  \lhead{\color{gray}\writeSubject}
  % \rfoot{Página \thepage}
  \cfoot{\thepage}
  % Eliminar línea horizontal del header
  % \renewcommand{\headrulewidth}{0pt}
  % Color de la línea y separación
  \renewcommand{\headrule}{\vspace{-5px}\hbox to\headwidth{\color{gray}\leaders\hrule height \headrulewidth\hfill}}
}


% Inserccion de imagen.
% \includeimg[descripción]{ruta}{anchura}
\newcommand{\includeimg}[3][]{
    \begin{figure}[H]
      \def\des{#1}
        \centering
        \includegraphics[width={#3}\textwidth]{{#2}}
        \ifx\des\empty
        \else
          \caption[0.1]{{#1}}
        \fi
    \end{figure}
}
